\documentclass[runningheads]{llncs}
\usepackage{biblatex} %Imports biblatex package
\usepackage{Packages}
\addbibresource{references.bib} %Import the bibliography file

\begin{document}
%
\title{Syntethische Generierung von Zeitreihendaten: Konzeptvorstellung von Generative Adversarial Networks (GAN) und TimeGAN}
%
\titlerunning{Syntethische Generierung von Zeitreihendaten: GAN und TimeGAN}
% If the paper title is too long for the running head, you can set
% an abbreviated paper title here
%
\author{Mario da Graca}
%
\authorrunning{M. da Graca}
% First names are abbreviated in the running head.
% If there are more than two authors, 'et al.' is used.
%
\institute{University of Applied Sciences Hamburg (HAW Hamburg)\\
Berliner Tor 5, 20999 Hamburg\\
}
%
\maketitle              % typeset the header of the contribution
%
\begin{abstract}
Trotz der Allgegenwärtigkeit von Zeitreihen, mangelt es für die Anwendung von Machine Learning Verfahren häufig an der nötigen Menge und Qualität, um zufriedenstellende Ergebnisse für Klassifikations, Forecastings oder Anomaly-Detection Problemstellungen zu erhalten. Die Generierung von synthetischen Zeitreihendaten soll diesem Problem Abhilfe schaffen. TimeGAN ist dabei ein generatives Modell, welches eindrucksvolle Ergebnisse erzielt, indem es die möglicherweise komplexen multivariate Charakteristiken einer Zeitreihe mithilfe eines Autoencoders reduziert und durch ein Generative Adversarial Networks (GAN) neue Daten generiert. GANs bestehen im grundlegenden aus zwei kontrahierenden neuronalen Netzen - dem Diskriminator und dem Generator. In einem an die Spieltheorie angelehnten Nullsummenspiel versucht der Generator den Diskriminator zu täuschen, indem er Daten generiert, die der Diskriminator nicht von realen Trainingsdaten unterscheiden kann.
Diese Arbeit soll ein tiefgehendes Verständnis über GANs, sowie dessen Anwendung auf Zeitreihendaten durch die Verwendung von TimeGAN, schaffen, um als Grundlage für zukünftige Arbeiten dienen zu können. Dazu wird der Aufbau, der mathematische Hintergrund und der Trainingsprozess des GAN-Frameworks beschrieben und im Speziellen auf das Trainingsschema des TimeGAN Modells eingegangen. Damit zukünftig generierte Zeitreihen bewertet werden können, werden außerdem Evaluationsmöglichkeiten durch qualitative und quantitative Methoden vorgestellt.

\keywords{Reinforcement Learning \and Curriculum Learning \and Unity ML-Agents}
\end{abstract}
%
%
%

%! Author = mario
%! Date = 25.01.2023

\section{Introduction}\label{sec:introduction}
Machine learning (ML) is a subset of Artificial Intelligence (AI) that focuses on the development of algorithms and statistical
models that allow computers to learn from data and make predictions or judgments without being explicitly programmed.
Reinforcement Learning is a type of Machine Learning that focuses on teaching agents to make decisions in a given environment
by maximizing a reward signal.
It is a method for an agent to learn how to behave in a given situation by executing specific
actions and watching the rewards that the environment provides.\\
ML and RL are widely used in a range of applications, such as robotics, autonomous vehicles, finance, healthcare and games.
This report explores Reinforcement Learning in the latter one, by creating a complex and agility focused labyrinth environment,
that an agent has to successfully solve.
With the help of Unity's ML-Agents~\cite{juliani2020} a model is trained using the standard Reinforcement Learning and Curriculum Learning~\cite{narvekar_learning_2018} approach.
The goal of this project was to explore the process of an agent learning a fine motor task.
The following chapter will explain the setup of the board game in real life and as a game in the Unity engine~\cite{Haas2014AHO} to lay the grounds
for the third chapter, where the reinforcement learning aspects of this project are described.
The focus is on the environment and training setup, as well as the reward engineering.
In chapter four the used metrics are discussed and the results presented, in order to give a qualified statement in chapter five
about the performances and show where the process can be improved.


%
% ---- Bibliography ----
%
% BibTeX users should specify bibliography style 'splncs04'.
% References will then be sorted and formatted in the correct style.
%
% \bibliographystyle{splncs04}
% \bibliography{mybibliography}
%
\printbibliography
%\bibliography{Quellen}

\end{document}
