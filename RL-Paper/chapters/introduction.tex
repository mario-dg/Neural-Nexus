%! Author = mario
%! Date = 25.01.2023

\section{Introduction}\label{sec:introduction}
Machine learning (ML) is a subset of Artificial Intelligence (AI) that focuses on the development of algorithms and statistical
models that allow computers to learn from data and make predictions or judgments without being explicitly programmed.
Reinforcement Learning is a type of Machine Learning that focuses on teaching agents to make decisions in a given environment
by maximizing a reward signal.
It is a method for an agent to learn how to behave in a given situation by executing specific
actions and watching the rewards that the environment provides.\\
ML and RL are widely used in a range of applications, such as robotics, autonomous vehicles, finance, healthcare and games.
This report explores Reinforcement Learning in the latter one, by creating a complex and agility focused maze environment,
that an agent has to successfully solve.
With the help of Unity's ML-Agents~\cite{juliani2020} a model is trained using the standard Reinforcement Learning and Curriculum Learning~\cite{narvekar_learning_2018} approach.
There are several goals defined for this project:
\begin{enumerate}
    \item Recreate the original wooden board game in Unity and set up the environment for training an agent.
    \item Train the agent on the whole maze and try to find the best performing set of hyperparameters and rewards to successfully
    follow the game-rule given path to the end, without falling into a hole and thus restarting the game.
    \item Train the agent using Curriculum Learning, where the agent starts of in an easier environment whose difficulty increases on successful completions.
    For this approach the environment has to be slightly modified compared to the previous task.
    \item Evaluate the performance of the agent on both tasks using the best set of hyperparameters found.
\end{enumerate}
The following chapter will explain basic concepts of Reinforcement Learning, Curriculum Learning and describes an overview
of important principles used in developing the application.
Chapter 3 goes into detail describing the environment setup, reward engineering process, agent configuration as well as metrics used
to analyze the agent's performance.
Finally the process and results are discussed to give an educated evaluation of the achievements and further optimizations.