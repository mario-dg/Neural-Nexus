% This is samplepaper.tex, a sample chapter demonstrating the
% LLNCS macro package for Springer Computer Science proceedings;
% Version 2.20 of 2017/10/04
%
\documentclass[runningheads]{llncs}
%
\usepackage{graphicx}
% Used for displaying a sample figure. If possible, figure files should
% be included in EPS format.
%
% If you use the hyperref package, please uncomment the following line
% to display URLs in blue roman font according to Springer's eBook style:
% \renewcommand\UrlFont{\color{blue}\rmfamily}

\begin{document}
%
    \title{Solving an agility maze using Reinforcement Learning with Unity ML-Agents and Curriculum Learning}
%
    \titlerunning{Comparing RL and Curriculum Learning using ML-Agents}
% If the paper title is too long for the running head, you can set
% an abbreviated paper title here
%
    \author{\textbf{Mario da Graca}}
%
    \authorrunning{M. da Graca}
% First names are abbreviated in the running head.
% If there are more than two authors, 'et al.' is used.
%
    \institute{
        \textsl{
            Self-optimizing Systems\\
            Winter Term 2022/23\\
            University of Applied Sciences (HAW Hamburg)\\
            Berliner Tor 5, 20999 Hamburg\\
        }
    }
%
    \maketitle              % typeset the header of the contribution
%
    \begin{abstract}
        This project report examines curriculum learning in Unity ML-Agents using a 3D agility maze task as an example.
        The task involves tilting the board on two axes to guide a marble through a maze, filled with walls and holes, through which the agent must steer the marble in order to follow the correct path and reach the end.
        The agent's performance was evaluated using two different approaches: standard reinforcement learning and curriculum learning.
        Curriculum learning involves starting the agent with simpler sub-tasks, and gradually increasing the difficulty over time.
        The paper covers the technical details of the project, including setup of the environment, tuning of parameters, reward engineering and performance analysis of the trained agent.

        \keywords{Reinforcement Learning \and Curriculum Learning \and Unity ML-Agents}
    \end{abstract}

    %! Author = mario
%! Date = 25.01.2023

\section{Introduction}\label{sec:introduction}
Machine learning (ML) is a subset of Artificial Intelligence (AI) that focuses on the development of algorithms and statistical
models that allow computers to learn from data and make predictions or judgments without being explicitly programmed.
Reinforcement Learning is a type of Machine Learning that focuses on teaching agents to make decisions in a given environment
by maximizing a reward signal.
It is a method for an agent to learn how to behave in a given situation by executing specific
actions and watching the rewards that the environment provides.\\
ML and RL are widely used in a range of applications, such as robotics, autonomous vehicles, finance, healthcare and games.
This report explores Reinforcement Learning in the latter one, by creating a complex and agility focused labyrinth environment,
that an agent has to successfully solve.
With the help of Unity's ML-Agents~\cite{juliani2020} a model is trained using the standard Reinforcement Learning and Curriculum Learning~\cite{narvekar_learning_2018} approach.
The goal of this project was to explore the process of an agent learning a fine motor task.
The following chapter will explain the setup of the board game in real life and as a game in the Unity engine~\cite{Haas2014AHO} to lay the grounds
for the third chapter, where the reinforcement learning aspects of this project are described.
The focus is on the environment and training setup, as well as the reward engineering.
In chapter four the used metrics are discussed and the results presented, in order to give a qualified statement in chapter five
about the performances and show where the process can be improved.

    \clearpage
    \bibliographystyle{splncs04}
    \bibliography{references}
%
\end{document}
