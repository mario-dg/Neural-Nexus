%! Author = mario
%! Date = 30.01.2023


\section{RL implementation}\label{sec:rl-implementation}

\subsection{Unity ML-Agents}\label{subsec:unity-ml-agents}
The Unity ML-Agents Toolkit is an open-source plugin for the Unity game engine.
It bundles easy to use functionality for environment creation, agent training, state-of-the-art RL algorithms and metric logging
all via the given Python API and Unity's editor interface~\cite{juliani2020}.
When working with this package there are only three main entities working together to build the training pipeline: Sensors, Agents and the Academy.
The Agent is the main component, it indicates that a game object serves as a trainable agent in the environment.
It can collect observations via different types of Sensors, take actions chosen by the policy and collect rewards to further improve its performance.
Possible types of observations are vectors of any length and any numerical datatype, images or ray-cast results.
The available RayPerceptionSensor casts rays from a specified point in any direction with an arbitrary length to determine distances to nearby game objects.
And lastly the Academy serves as the connection point, as it manages the simulation, its global steps and the agents in the environment.
The Academy can also be used to speed up the training process, by multiplying the training arenas in the scene, allowing a parallel training of agents on the same policy.
An experience buffer of specified size is filled up by all agents and once it's full the policy is updated.

\subsection{Basics}\label{subsec:basics}
The reason of this project is to compare the default reinforcement learning approach with the idea of curriculum learning.
Very hard or enduring tasks might take a long time for an agent to solve, hence the idea of splitting it into smaller subtasks seems obvious.
Curriculum learning is an extension to the widely used transfer learning in other Machine Learning areas~\cite{wiering_transfer_2012}.
Its goal is to create a series of source tasks for the agent to train on, such that the acquired knowledge
can be transferred to increase the learning speed and performance on the target task~\cite{narvekar_learning_2018}.
The major obstacle in using curriculum learning is building and choosing the subtasks~\cite{narvekar2016source}.
For this specific game three smaller sub-levels were created of the labyrinth~\cite{fig:unity_levels}.
The main idea is to train the agent on the whole labyrinth without holes first, to ensure that it is able to follow the specified path reliably.
In the following stages more and more holes are introduced, such that only small adjustments to the movements have to be made.
An agent advances to the next subtask, when it is able to solve the current task 100 times.

\begin{figure}[h]
    \centering
    \caption{Four stages of the labyrinth used for curriulum learning.}
    \label{fig:unity_levels}
\end{figure}

To train the agent the Proximal Policy Optimization (PPO)~\cite{schulman_proximal_2017} algorithm is used.
Since it's an on-policy algorithm that aims to find the best balance between exploration and exploitation, it's especially
well suited for this task in a continuous environment.
The alternative ML-Agents offer is is Soft-Actor Critic (SAC)~\cite{haarnoja_soft_2018}.
This approach is not used, as it is better suited for a sparse reward system, higher dimensional action space and heavier or slower
environments, with about 0.1s between steps, as suggested by Unity.

\subsection{Environment}\label{subsec:environment}

\subsubsection{Action Space}
The action space refers to the set of possible actions an agent can take in a given environment.
It is usually represented as a discrete or continuous space, where discrete actions are chosen as distinct elements
out of a finite set, whereas continuous actions are real-valued vectors, and in Unity's case are clipped between $[-1, 1]$.
As the controls of this game are fairly simple, so is the action space of this reinforcement learning task.
The size of the action space is set to two floats, meaning the Python API fills a buffer with two floats ranging between $[-1, 1]$ that will be converted into
the corresponding controls:
\begin{enumerate}
    \item{Rotation on X-Axis}: Rotates the labyrinth on the x-axis by the given amount in the action buffer multiplied with a constant rotation factor of 0.1.
    Depending on the sign of the action the board is either tilted towards positive or negative x.
    \item{Rotation on Z-Axis}: Rotates the labyrinth equivalent on the z-axis.
    Unity uses a left-handed, y-up coordinate system.
\end{enumerate}
Both actions are disconnected, meaning in one step the labyrinth can be tilted both on the x- and z-axis simultaneously.
By using a continuous action space the agent can tilt the game board more precisely, which results in finer control over the marble.

\subsubsection{Observation Space}
Contrary to the action space, the observation space describes the input of the agent, information that the agent receives from the environment.
It can also be discrete or continuous and is used to inform the decision making process of the agent.
The difficult trade-off is giving the agent enough data about it's surrounding to be able to converge on a good solution and not giving
too much irrelevant data, which slows down the learning process or could lead to suboptimal performances.
Agents in the ML-Agents package distinguish between two types of observations: vectors and images.
Images can be of arbitrary size with either grayscale or color information and vectors can be of any dimension with numerical values.
As mentioned above observations can also be collected off of Sensors like the RayPerceptionSensor, which yields information about
a ray-hit with another game object, like the normalized distance.
The best results were achieved with following set of observations:
\begin{itemize}
    \item{Rotation on X-Axis}: 4 floats represented as a quaternion
    \item{Rotation on Z-Axis}: 4 floats represented as a quaternion
    \item{Normalized Position of the marble}: 3 floats
    \item{Magnitude of Marble Velocity}: 1 float
    \item{Normalized Euclidian Distance to the next Checkpoint}: 1 float
    \item{RayPerceptionSensor Checkpoints}: A RayPerceptionSensor with 8 total rays around the marble to detect the next checkpoint
    \item{RayPerceptionSensor Holes}: A RayPerceptionSensor with 8 total rays around the marble tilted downwards to detect holes
\end{itemize}
This results in 13 float values passed in manually into the observation space buffer, with two added RayPerceptionSensors (16 rays)
automatically added by Unity.
Defining the observation space is a crucial step in developing a reinforcement learning game~\cite{song_observational_2019}.

\subsubsection{Reward Engineering}
Another very important step that determines the success of the learning process is the reward engineering~\cite{sutton_introduction_1992}.
Reward engineering describes the process of defining and shaping the reward signal, which is a scalar value that provides
feedback to the agent about its performance.
It indicates whether the actions it takes are good or bad.
A good reward function must accurately reflect the desired behavior of the agent, otherwise suboptimal or even
unwanted behaviors, that do not align with the intended task, can be the pursued by the agent.
There are two different ways of rewarding an agent in Unity's ML-Agents: explicit rewards and passive rewards.
With explicit rewards or punishments specific events are rewarded, meanwhile passive rewards encourage or discourage certain behaviors.
Both techniques are used for the skill labyrinth and are identical for both approaches, curriculum learning and normal learning.
During testing this step has gone through many iterations with vastly different reward functions.
The following set of rewards and punishments performed best in the scope of testing time in this project.\\\\
Positive rewards:
\begin{enumerate}
    \item Reaching the next checkpoint in line adds (+XXX * index of the checkpoint), encourages the agent to stick to the given path.
    \item To ensure that the agent doesn't steer the marble around too much, resulting in falling into holes or going the wrong way, the agent
    is reward the closer it gets to the next checkpoint (TODO: Formel).
    \item Reaching the goal adds (+1000.0)
\end{enumerate}
Negative rewards:
\begin{enumerate}
    \item Falling into a hole sets (-XXX). Ends the episode with a negative reward.
    \item Going through a wrong checkpoint sets (-XXX). End the episode with a negative reward.
    \item Analogue to the reward the agent is punished, if the marble is moving away from the next checkpoint or not moving at all.
    Adds (-XXX TODO:Formel)
    \item If the marbles velocity is below a certain threshold (XXX) for more than XXX steps, the episode resets with a negative reward (-XXX)
\end{enumerate}
This set of punishments and rewards was chosen based on the rules of the board game and the intended goal of the game:
Following the path to reach the end without falling into a hole.

\subsubsection{Hyperparameters}
Hyperparameters are parameters that are set before training and can not be learned from the data.
Unity ML-Agents offers a great flexibility by making all hyperparameters customisable.
This complicates the training process, since hyperparameter tuning is the main task in any Machine Learning application.
By observing the results and the training process, those parameters can be adjusted.
Doing this over and over again is a time consuming task, but it is necessary in order to achieve good results.
The following hyperparameters were tuned the most.\\
\textbf{batch\_size:}\\
The number of experiences (i.e.\ samples) to be used in each iteration of the training process.
Larger batch sizes can lead to faster convergence, but can also lead to slower training due to increased memory usage.\\
\textbf{buffer\_size:}\\
The maximum number of experiences to store in the experience replay buffer.
The agent samples from this buffer to learn from past experiences.
If the buffer size is too small, the agent may not have enough experiences to learn from.
If the buffer size is too large, the agent may waste memory on irrelevant experiences.\\
\textbf{lambd:}\\
A parameter used in the Generalized Advantage Estimation (GAE) algorithm~\cite{schulman_high-dimensional_2015}, which is used to calculate the advantages of actions in reinforcement learning.
The value of lambda determines the trade-off between bias and variance in the estimation of the advantages.\\
\textbf{hidden\_units:}\\
The number of neurons in the hidden layers of the neural network used for the policy and value function approximations.
Increasing the number of hidden units can result in a more complex model that fits the data better, but also increases the risk of overfitting.\\
\textbf{num\_layers:}\\
The number of hidden layers in the neural network used to approximate the policy and value functions.
Adding more hidden layers can increase the capacity of the model, allowing it to fit more complex relationships in the data.
However, adding too many hidden layers can lead to overfitting and slow down the training process.\\

\begin{center}
    \rowcolors{2}{gray!25}{white}
    \begin{tabular}{|c|c|c|}
        \rowcolor{gray!50}
        \hline
        \textbf{Hyperparameter} & \textbf{Normal RL} & \textbf{Curriculum Learning} \\ \hline
        batch\_size    & 2048      & 2048                \\ \hline
        buffer\_size   & 40960     & 40960               \\ \hline
        lambd          & 0.99      & 0.98                \\ \hline
        hidden\_units  & 1024      & 1024                \\ \hline
        num\_layers    & 3         & 3                   \\ \hline
    \end{tabular}
\end{center}
\begin{center}
    \vspace{10pt}
    \textbf{Tab. 1.} Used hyperparameters for normal RL approach and Curriculum Learning.
\end{center}
For the rest of the hyperparameters the default values were used.

\subsubsection{Training}
The concepts of reproducibility and determinism are critical in physics engines because they determine the accuracy and predictability of simulation results.
Reproducibility in a physics engine refers to the ability to obtain the same simulation results when run multiple times under the same conditions.
Determinism refers to the fact that, regardless of the hardware or software environment, simulation results should always be the same for the same inputs.
In the Unity game engine both concepts can't be fully guaranteed.
Trying to compare reinforcement learning with curriculum learning can't happen under fair circumstances, therefore both
approaches are run five times with the best set of hyperparameters, to compare a statistical average.
To reduce training time, the maximum number of steps was set to 75 million and concepts like parallel training were applied.



